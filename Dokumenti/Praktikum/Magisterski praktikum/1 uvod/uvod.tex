% !TeX spellcheck = si_SI
\chapter{Uvod}\label{cha:uvod}
\section{Ozadje problema}\label{sec:ozadje_problema}
    
    Prisotnost nenadzorovanih vibracij v inženirskih aplikacijah hitro vodi v prekomerno obrabo, poškodbe ali celo v kritično in nevarno odpoved materiala ter strojnih delov. Z implementacijo vibroizolacije lahko omogočimo zaščito opazovanega sistema pred zunanjim povzročiteljem dinamičnih sil. 
    
    V praksi je največkrat uporabljena linearna vibroizolacija, ki je učinkoviti le, če je njena lastna frekvenca precej nižja od frekvence vzbujanja. Za odpravo te pomanjkljivosti je bila zasnovana nelinearna vibroizolacija z visoko statično in nizko dinamično (VSND) togostjo, ki izolira vibracije v nizkofrekvenčnem območju. Pri tem visoka statična togost pomeni veliko nosilnost, medtem ko nizka dinamična togost pomeni povečano območje izolacije pri nizkih frekvencah. Skoraj ničelna dinamična togost v delovni točki, ki je značilna za nelinearno vibroizolacijo, se imenuje kvazi ničelna togost (KNT), nelinearni izolatorji z značilnostmi KNT pa se v nadaljevanju imenujejo kvazi ničelni togostni izolatorji ali KNT izolatorji.
    
    Metamateriali (MM) so v zadnjih dveh desetletjih deležni vse večjega raziskovalnega zanimanja, saj imajo zanimive optične, elektromagnetne, akustične in mehanske lastnosti. Znani tudi pod imenom metastrukture, izražajo posebne fizikalne lastnosti, ki jih v naravi ne najdemo. Gre za serijo ponavljajočih se in geometrijsko skrbno načrtovanih podstruktur. Podobni so celičnim strukturam, njihove posebne fizikalne lastnosti, ki jih v naravi ne najdemo, pa je mogoče preučevati iz osnovne reprezentativne celice (ROC). Vibroizolativni MM sestoji iz periodično razporejenih lokalnih resonatorjev, ki absorbirajo vibracijsko energijo v določenem frekvenčnem območju. 
    
    Zaradi prostorske razporeditve osnovne celice se je aditivna tehnologija izkazala kot odlični proces za izdelavo metastruktur. Geometrijsko kompleksne podstrukture lahko skoraj poljubno postavimo v prostoru in jih na makronivoju formuliramo v MM poljubne oblike.

    Rešitev s pasivno vibroizolacijo ima prednost enostavnosti rešitve, brez kompleksnih mehanizmov, ki bi potrebovali zunanje napajanje ali krmiljenje. Negativna posledica je nezmožnost prilagajanja na spremembe delovnih pogojev stroja ali okolice. Rešitev predstavlja semi-pasivna vibroizolacija, ki omogoča regulacijo in prilagajanje na spremembe. 


\section{Cilji praktikuma}\label{sec:cilji_naloge}

    Osrednja problematika naloge je raziskava dinamike 3D tiskanih termoaktivnih metamaterialnih KNT vibroizolatorjev za uporabo v nizkofrekvenčnem področju.
    
    V prvem delu naloge predstavimo MM in nato potrebne teoretične osnove za formulacijo reprezentativne osnovne celice (ROC) MM, ki izkazuje VSND togost ali kvazi ničelno togost (KNT). Bistveno vlogo za doseganje KNT lastnosti nelinearnih vibroizolatorjev igrajo elementi z negativno togostjo (NT) imajo ključno vlogo pri oblikovanju funkcije KNT, saj nevtralizirajo pozitivno togostjo (PT) struktur. Značilnost NT se lahko uresniči iz različnih struktur, kot so poševne vzmeti, upogibni nosilci, magnetne vzmeti in biološko navdihnjene strukture. Pri metamaterialih lahko NT lastnost dosežemo z vključitvijo bistabilne strukture, ki ob preskoku iz enega v drugo stabilno stanje nasprotuje PT strukture.
    
    Večje število zaporednih ROC tvori enodimenzionalni KNT MM. Za dinamično analizo MM vibroizolatorja je opisana teorija dvomasnega dušilca nihanj. Z obravnavo MM kot neskončne periodične strukture lahko analitično določimo frekvenčno pasovno vrzel, v kateri je odziv sistema manjši. 
    
    Analitični popisi realnih struktur imajo svoje omejitve, saj je natančen popis geometrije nemogoč. S tem razlogom bomo dinamiko problema rešili z metodo končnih elementov (MKE). 
    
    Opišemo tudi teorijo MM, ki je izdelan z uporabo aditivne tehnologije, pri čemer uporabimo filament iz polilaktične kisline (PLA - \textit{Polylactic acid}) z aditivom grafitnega prahu - torej GPLA (\textit{Graphite Polylactic acid}), kar omogoča prevajanje elektrike skozi MM. Z uporovnim segrevanjem materiala preko Joulovega toka bomo dosegli njegovo mehčanje in zmanjšanje togosti, ki je bistvena lastnost pri obravnavi nihanja. Tako dosežemo adaptivni metamaterial, ki ima v primerjavi z pasivnimi vibroizolatorji prilagodljivo in posledično širše uporabno frekvenčno območje. Na GPLA vzorcih izmerimo potrebne materialne lastnosti za konkretno določitev parametrov ROC.
        
    Zadnji korak, ki bo del magistrske naloge, je meritev odziva na dinamične motnje v okolici lastnih frekvenc in vpliv izdelane MM vibroizolacije. Tukaj bomo z spreminjanjem delovnih pogojev želeli zmanjšati učinkovitost metamateriala, ki pa se bo zaradi svoje adaptivnosti bil zmožen novim parametrom prilagoditi.


