% !TeX spellcheck = si_SI
\addifeven

\vspace*{0mm}
\Large\textbf{Seznam uporabljenih simbolov}\normalsize
\vskip\medskipamount
\leaders\vrule width \textwidth\vskip0.4pt
\vskip\bigskipamount

\addcontentsline{toc}{section}{Seznam uporabljenih simbolov}

\begin{longtable}[l]{@{}p{.15\textwidth}@{}p{.15\textwidth}@{}p{.7\textwidth}@{}}
\hline
Oznaka & Enota & Pomen\\
\hline
\endhead
\hline
\endfoot
\hline
\endlastfoot
    $A$     & $\text{m}^2$          & površina preseka\\
    $b$     & m                     & globina nosilca\\
    $c$     & $\text{N s}^2/\text{m}$, J/kg K & koeficient dušenja, specifična toplota \\
    $d$, $\Delta$ & m, / & (brezdimenzijska) dolžina pomika kosinusnega nosilca \\ 
    $e$     & / & Eulerjevo število $2,71828...$ \\
    $E$     & Pa, J                 & modul elastičnosti ali energija\\
    $f$     & /, Hz & brezdimenzijska sila ali lastna frekvenca \\
    $F$     & N                     & sila \\
    $h$     & m, W/mK               & višina kosinusnega loka, toplotna prestopnost \\
    $i$     & /                     & imaginarno število $\sqrt{-1}$ \\
    $I$     & $\text{m}^4$, A       & vztrajnostni moment prereza, električni tok\\
    $k$     & N/mm                  & (nelinearna) togost\\
    $K$     & N/mm                  & linearna togost ROC\\
    $l, L$  & m                     & dolžina\\
    $\mathcal{L}$ & / & Lagrangian \\
    $m$ & kg & masa (resonatorja)\\
    $M$ & N m,  kg & moment ali  masa ROC \\
    $N$  & / & brezdimenzijska aksialna sila \\
    $P$    & N, W & sila ali moč \\
    $q$, $Q$ & /, m & (brezd.) relativna pr. stopnja ali vozliščni pomiki MKE\\ 
    $R$     & m, /,  $\Omega$           & radij nosilca, brezdimenzijska  lateralna sila ali el. upornost\\
    $s$, $S$ & m, / & (brezdimenzijska) dolžina aksialnega nosilca \\
    $t$     & m, s                  & globina nosilca, čas \\
    $T$ & $^\text{o}$C, / & temperatura ali prenosna funkcija \\ 
    $u$, $U$ & /, J, m, V & (normalizirana) energija, pr. stopnja ali el. napetost \\
    $v$, $V$ & /, m, m$^3$ & (brezdimenzijska) pr. stopnja resonatorja, volumen \\
    $w$, $W$ & / & (brezdimenzijski) pomik (kosinusnega nosilca)\\
    $\overline{w}$, $\overline{W}$ & m, / & (normirana) prva deformacijska oblika kosinusnega nosilca\\
    $x$, $X$ & /, m & (brezdimenzijska) prostostna stopnja \\

    \hline
    $[\,A\,]$   & /& dinamska matrika sistema \\
    $[\,B\,]$   & / & matrika odvodov oblikovnih funkcij \\
    $[\,C\,]$       & N $\text{s}^2$/m & disipacijska matrika\\
    $[\,D\,]$   & / & materialna matrika \\
    $[\,I\,]$   & /& identiteta \\
    $[\,J\,]$   & /& Jakobijeva matrika \\
    $[K]$       & Pa $\text{m}^2$/m & togostna matrika\\
    $[M]$       & kg & masna matrika\\
    $[\,N\,]$   & / & aproksimacijske funkcije \\
    $[\,T\,]$   &/ & transformacijska matrika \\
    

    &&\\
    $\alpha$ & / & brezdimenzijska togost   \\
    $\beta$ & /  &  brezdimenzijska masa \\
    $\gamma$ & / & razmerje dolžin \\
    $\delta$ & / &  koeficient brezdimenzijske sile tretje potence \\
    $\delta F$  & N & virtualna sila \\
    $\delta M$  & N m & virtualni moment\\
    $\delta W$  & J  & virtualno delo \\
    $\epsilon$ & / & brezdimenzijski red amplitude gibanja \\
    $\varepsilon$ & / & specifična deformacija \\
    $\zeta$ & /  & razmernik dušenja  \\
    $\eta$ & / &  koeficient brezdimenzijske sile pete potence \\
    $\theta$ & rad & kot horizontalne vzmeti \\ 
    $\kappa$ & 1/rad  &  krožna frekvenca resonatorja \\
    $\varkappa$ & / & razmerje krožnih frekvenc  \\ 
    $\lambda$ & / & lastne vrednosti \\
    $\mu$ & / & razmerje togosti ali disperzijska krivulja \\
    $\nu$ &  & Poissonov količnik \\
    $\pi$     & / & konstanta $3,1416...$ \\
    $\rho$  & kg/$\text{m}^2$   & gostota\\
    $\varrho$ & $\Omega$ mm$^2$/m & specifična električna upornost \\
    $\sigma$ & Pa & napetost \\
    $\tau$ & / & brezdimenzijski čas \\
    $\phi$ & rad, F & prostostna stopnja kota \\ 
    $\overline{\phi}$ & F & površinske sile \\
    $\overline{\Phi}$ & F & volumske sile \\
    $\omega$ & 1/rad  & krožna frekvenca ROC-ja\\
    $\Omega$ & / &  razmerje frekvence z lastno frekvenco ROC \\
    
    


\end{longtable}


\newpage
\begin{longtable}[l]{@{}p{.15\textwidth}@{}p{.85\textwidth}@{}}
\hline
Indeksi & \\
\hline
\endfirsthead
\hline
\endhead
    0 & začetno stanje, lastna frekvenca \\
    I,1 & 1. polje 1. glavnega sistema\\
    II,1 & 2. polje 1. glavnega sistema\\
    I,2 & 1. polje 2. glavnega sistema\\
    II,2 & 2. polje 2. glavnega sistema\\
    A & točka A\\
    b & upogibno \\
    B & točka B\\
    c & koncentrirano \\
    d & disperzijsko \\
    f & aktuacijsko \\
    g & steklenje \\
    h & horizontalno \\
    k & kinetično, končno \\
    KNT & kvazi ničelna togost \\
    MNT & mehanizem negativne togosti \\
    n & negativna\\
    $n1$ & negativna togost prvega območja \\
    $n2$ & negativna togost drugega območja \\
    $n3$ & negativna togost tretjega območja \\
    N & notranje \\
    opt & optimalno \\
    p & pozitivno, potencialno, konstanten tlak\\
    ROC & reprezentativna osnovna celica \\
    s, S & tlačno, površinsko \\
    sd & standardna deviacija \\
    sp & spodnja \\
    sr & srednja \\
    t & celotno \\
    v & vertikalno, volumsko \\
    Z & zunanje \\
    zg & zgornja \\
    $\infty$ & neskončno oddaljeno \\

\end{longtable}

\newpage

\addifeven

\vspace*{0mm}
\Large\textbf{Seznam uporabljenih okrajšav}\normalsize
\vskip\medskipamount
\leaders\vrule width \textwidth\vskip0.4pt
\vskip\bigskipamount

\addcontentsline{toc}{section}{Seznam uporabljenih okrajšav}

\begin{longtable}[l]{@{}p{.2\textwidth}@{}p{.8\textwidth}@{}}
\hline
Okrajšava & Pomen\\
\hline
\endfirsthead
\hline
\endhead
&\\
    3D & tridimenzionalno \\
    DVB & dinamični vibracijski blažilec \\
    FPF & frekvenčna prenosna funkcija \\
    GPLA & polilaktična kislina z aditivom grafitnega prahu \\ 
         &(ang. \emph{Graphite Polylactic Acid})\\
    KE & končni element \\
    KNT & kvazi ničelna togost \\
    KS & koordinatni sistem \\
    MKE & metoda končnih elementov \\
    MM & metamateriali \\
    MNT & mehanizem negativne togosti \\
    NT & negativna togost \\
    PLA & polilaktična kislina (ang. \emph{Polylactic acid})\\
    PT & pozitivna togost \\
    PZF & pasovno zavrnitveni filter \\
    ROC & reprezentativna osnovna celica \\
    VSND & visoka statična in nizka dinamična \\
\end{longtable}

\newpage
